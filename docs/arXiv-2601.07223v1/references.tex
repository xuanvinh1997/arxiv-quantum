%%%%%%%%%%%%%%%%%%%%%%%% referenc.tex %%%%%%%%%%%%%%%%%%%%%%%%%%%%%%
% sample references
% %
% Use this file as a template for your own input.
%
%%%%%%%%%%%%%%%%%%%%%%%% Springer-Verlag %%%%%%%%%%%%%%%%%%%%%%%%%%
%
% BibTeX users please use


\bibliographystyle{unsrt}
\bibliography{references}
%
% \biblstarthook{References may be \textit{cited} in the text either by number (preferred) or by author/year.\footnote{Make sure that all references from the list are cited in the text. Those not cited should be moved to a separate \textit{Further Reading} section or chapter.} If the citation in the text is numbered, the reference list should be arranged in ascending order. If the citation in the text is author/year, the reference list should be \textit{sorted} alphabetically and if there are several works by the same author, the following order should be used:
% \begin{enumerate}
% \item all works by the author alone, ordered chronologically by year of publication
% \item all works by the author with a coauthor, ordered alphabetically by coauthor
% \item all works by the author with several coauthors, ordered chronologically by year of publication.
% \end{enumerate}
% The \textit{styling} of references\footnote{Always use the standard abbreviation of a journal's name according to the ISSN \textit{List of Title Word Abbreviations}, see \url{http://www.issn.org/en/node/344}} depends on the subject of your book:
% \begin{itemize}
% \item The \textit{two} recommended styles for references in books on \textit{mathematical, physical, statistical and computer sciences} are depicted in ~\cite{science-contrib, science-online, science-mono, science-journal, science-DOI} and ~\cite{phys-online, phys-mono, phys-journal, phys-DOI, phys-contrib}.
% \item Examples of the most commonly used reference style in books on \textit{Psychology, Social Sciences} are~\cite{psysoc-mono, psysoc-online,psysoc-journal, psysoc-contrib, psysoc-DOI}.
% \item Examples for references in books on \textit{Humanities, Linguistics, Philosophy} are~\cite{humlinphil-journal, humlinphil-contrib, humlinphil-mono, humlinphil-online, humlinphil-DOI}.
% \item Examples of the basic Springer Nature style used in publications on a wide range of subjects such as \textit{Computer Science, Economics, Engineering, Geosciences, Life Sciences, Medicine, Biomedicine} are ~\cite{basic-contrib, basic-online, basic-journal, basic-DOI, basic-mono}. 
% \end{itemize}
% }

%\begin{thebibliography}{99.}%
% and use \bibitem to create references.
%
% Use the following syntax and markup for your references if 
% the subject of your book is from the field 
% "Mathematics, Physics, Statistics, Computer Science"
%
% Contribution
%\bibitem{QVC} Mitarai, Kosuke and Negoro, Makoto and Kitagawa, Masahiro and Fujii, Keisuke.: Quantum circuit learning, Physical Review A, \textbf{98}, 032309 (2018)

%\bibitem{surface_codes} Fowler, Austin G. et. al.: Surface codes: Towards practical large-scale quantum computation, Physical Review A, \textbf{86}, 032324 (2012)

%\bibitem{magic_state_distillation_cost1}
%Litinski, Daniel.: A game of surface codes: Large-scale quantum computing with lattice surgery, Quantum \textbf{3}, 128 (2019).

%\bibitem{magic_state_distillation_cost2}
%Litinski, Daniel.: Magic state distillation: Not as costly as you think, Quantum \textbf{3}, 205 (2019).

%\bibitem{Azure_quantum_resource_estimator}
%\url{https://github.com/microsoft/qsharp/tree/main/resource_estimator}

%\bibitem{Azure_quantum_resource_estimator_paper}
%Beverland, Michael E et. al.: Assessing requirements to scale to practical quantum advantage, arXiv preprint arXiv:2211.07629 (2022)

%\bibitem{local_cost_function_1}
%Cerezo, Marco et al.: Cost function dependent barren plateaus in shallow parametrized quantum circuits, Nature Communications \textbf{12}, 1791 (2021).

%\bibitem{noisy_classical_CNN}
%K. Audhkhasi, O. Osoba, and B. Kosko.: Noise-enhanced
%convolutional neural networks, Neural Networks \textbf{78}, 15
%(2016)

%\bibitem{stochastic_classical_NN}
%C. Jin, P. Netrapalli, R. Ge, S. M. Kakade, and M. I.
%Jordan.: On nonconvex optimization for machine learning:
%Gradients, stochasticity, and saddle points, Journal of the
%ACM (JACM) \textbf{68}, 1 (2021)

%\bibitem{MNIST}
%L. Deng.: The mnist database of handwritten digit images
%for machine learning research [best of the web], IEEE
%signal processing magazine \textbf{29}, 141 (2012).

%\bibitem{T_consumption_for_unitary_improved}
%C. Gidney.: How to factor 2048 bit RSA integers
%with less than a million noisy qubits, arXiv preprint
%arXiv:2505.15917 (2025).

%\bibitem{dynamical_decoupling}
%Uhrig, G{\"o}tz S, Keeping a quantum bit alive by optimized $\pi$-pulse sequences, Phys. Rev. Lett. \textbf{98}, 100504 (2007).

%\bibitem{pauli_twirling_1}
%Wallman, Joel J and Emerson, Joseph, Noise tailoring for scalable quantum computation via randomized compiling, Physical Review A \textbf{94}, 052325 (2016).

%\bibitem{pauli_twirling_2}
%Van Den Berg, Ewout et. al., Probabilistic error cancellation with sparse Pauli--Lindblad models on noisy quantum processors, Nature physics \textbf{19}, 1116--1121 (2023).

%\bibitem{zero_noise_extrapolation}
%Giurgica-Tiron, Tudor et. al., Digital zero noise extrapolation for quantum error mitigation, 2020 IEEE International Conference on Quantum Computing and Engineering (QCE), 306--316, (2020).

%\bibitem{qec_lattice_surgery}
%A. G. Fowler and C. Gidney.: Low overhead quantum computation using lattice surgery, arXiv preprint
%arXiv:1808.06709 (2018)

%\bibitem{noise_induced_barren_plateaus}
%S. Wang et al.: Noise-induced barren plateaus in variational quantum algorithms, Nature Communications \textbf{12}, 6961 (2021).

%\bibitem{shot_noise_avoids_saddlepoints}
%J. Liu, F. Wilde, A. A. Mele, X. Jin, L. Jiang, and J. Eis-
%ert.: Stochastic noise can be helpful for variational quantum algorithms, Physical Review A \textbf{111}, 052441 (2025).

%\bibitem{physical_T_and_logical_T_error_rate}
% Y. Li.: A magic state’s fidelity can be superior to the
%operations that created it, New Journal of Physics \textbf{17},
%023037 (2015).

%\bibitem{SI1000}
%C. Gidney, M. Newman, A. Fowler, and M. Broughton.:
%A fault-tolerant honeycomb memory, Quantum \textbf{5}, 605
%(2021).
%\bibitem{randomised_benchmarking}
%E. Magesan, J. M. Gambetta, and J. Emerson.: Scalable
%and robust randomized benchmarking of quantum pro-
%cesses, Phys. Rev. Lett. \textbf{106}, 180504 (2011).

%\bibitem{chamberland2020very}
%C. Chamberland and K. Noh.: Very low overhead fault-tolerant magic state preparation using redundant ancilla
%encoding and flag qubits, npj Quantum Information \textbf{6},
%91 (2020).

%\bibitem{T_consumption_for_unitary}
%Kliuchnikov, Vadym et. al., Shorter quantum circuits via single-qubit gate approximation, Quantum \textbf{7}, 1208 (2023).

% \bibitem{science-contrib} Broy, M.: Software engineering --- from auxiliary to key technologies. In: Broy, M., Dener, E. (eds.) Software Pioneers, pp. 10-13. Springer, Heidelberg (2002)
%
% % Online Document
% \bibitem{science-online} Dod, J.: Effective substances. In: The Dictionary of Substances and Their Effects. Royal Society of Chemistry (1999) Available via DIALOG. \\
% \url{http://www.rsc.org/dose/title of subordinate document. Cited 15 Jan 1999}
% %
% % Monograph
% \bibitem{science-mono} Geddes, K.O., Czapor, S.R., Labahn, G.: Algorithms for Computer Algebra. Kluwer, Boston (1992) 
%
% Journal article
% \bibitem{science-journal} Hamburger, C.: Quasimonotonicity, regularity and duality for nonlinear systems of partial differential equations. Ann. Mat. Pura. Appl. \textbf{169}, 321--354 (1995)

%\bibitem{Vaidman_1996} Vaidman, L., Goldenberg, Lior., Wiesner, S.: Error prevention scheme with four particles. Physical Review A, \textbf{54}, 3, R1745--R1748 (1996)
% doi: \href{https://doi.org/10.1103/PhysRevA.54.R1745}{10.1103/PhysRevA.54.R1745}.

%\bibitem{Arute2019} Arute, F., et al.: Quantum supremacy using a programmable superconducting processor. Nature, \textbf{574}, 7779, 505--510 (2019)

%\bibitem{Pascuzzi_2022} Pascuzzi, V. R., He, A., Bauer, C., W., de Jong, W., A., and Nachman, B.: Computationally efficient zero-noise extrapolation for quantum-gate-error mitigation. Physical Review A, \textbf{105}, 4, 2469--9934 (2022)

%\bibitem{PhysRevA.80.032314} Khodjasteh, K., and Viola, L.: Dynamical quantum error correction of unitary operations with bounded controls. Phys. Rev. A, \textbf{80}, 3, 032314 (2009)

%\bibitem{PhysRevX.10.011022} Chamberland, C., Zhu, G., Yoder, T., J.,  Hertzberg, J., B., and Cross, A., W.: Topological and Subsystem Codes on Low-Degree Graphs with Flag Qubits. Phys. Rev. X, \textbf{10}, 1, 011022 (2020)

%\bibitem{PRXQuantum.1.010302} Chao, R., and Reichardt, B., W.: Flag Fault-Tolerant Error Correction for any Stabilizer Code. PRX Quantum, \textbf{1}, 1, 010302 (2020)





%
% % Journal article by DOI
% \bibitem{science-DOI} Slifka, M.K., Whitton, J.L.: Clinical implications of dysregulated cytokine production. J. Mol. Med. (2000) doi: 10.1007/s001090000086 
%
% \bigskip

% Use the following (APS) syntax and markup for your references if 
% the subject of your book is from the field 
% "Mathematics, Physics, Statistics, Computer Science"
%
% Online Document
% \bibitem{phys-online} J. Dod, in \textit{The Dictionary of Substances and Their Effects}, Royal Society of Chemistry. (Available via DIALOG, 1999), 
% \url{http://www.rsc.org/dose/title of subordinate document. Cited 15 Jan 1999}
% %
% % Monograph
% \bibitem{phys-mono} H. Ibach, H. L\"uth, \textit{Solid-State Physics}, 2nd edn. (Springer, New York, 1996), pp. 45-56 
% %
% % Journal article
% \bibitem{phys-journal} S. Preuss, A. Demchuk Jr., M. Stuke, Appl. Phys. A \textbf{61}
% %
% % Journal article by DOI
% \bibitem{phys-DOI} M.K. Slifka, J.L. Whitton, J. Mol. Med., doi: 10.1007/s001090000086
%
% Contribution 
% \bibitem{phys-contrib} S.E. Smith, in \textit{Neuromuscular Junction}, ed. by E. Zaimis. Handbook of Experimental Pharmacology, vol 42 (Springer, Heidelberg, 1976), p. 593
%
% \bigskip
%
% Use the following syntax and markup for your references if 
% the subject of your book is from the field 
% "Psychology, Social Sciences"
%
%
% Monograph
% \bibitem{psysoc-mono} Calfee, R.~C., \& Valencia, R.~R. (1991). \textit{APA guide to preparing manuscripts for journal publication.} Washington, DC: American Psychological Association.
% %
% % Online Document
% \bibitem{psysoc-online} Dod, J. (1999). Effective substances. In: The dictionary of substances and their effects. Royal Society of Chemistry. Available via DIALOG. \\
% \url{http://www.rsc.org/dose/Effective substances.} Cited 15 Jan 1999.
% %
% % Journal article
% \bibitem{psysoc-journal} Harris, M., Karper, E., Stacks, G., Hoffman, D., DeNiro, R., Cruz, P., et al. (2001). Writing labs and the Hollywood connection. \textit{J Film} Writing, 44(3), 213--245. 

%
% Contribution 
% \bibitem{psysoc-contrib} O'Neil, J.~M., \& Egan, J. (1992). Men's and women's gender role journeys: Metaphor for healing, transition, and transformation. In B.~R. Wainrig (Ed.), \textit{Gender issues across the life cycle} (pp. 107--123). New York: Springer.
% %
% % Journal article by DOI
% \bibitem{psysoc-DOI}Kreger, M., Brindis, C.D., Manuel, D.M., Sassoubre, L. (2007). Lessons learned in systems change initiatives: benchmarks and indicators. \textit{American Journal of Community Psychology}, doi: 10.1007/s10464-007-9108-14.
%
%
% Use the following syntax and markup for your references if 
% the subject of your book is from the field 
% "Humanities, Linguistics, Philosophy"
%
% \bigskip
%
% Journal article
% \bibitem{humlinphil-journal} Alber John, Daniel C. O'Connell, and Sabine Kowal. 2002. Personal perspective in TV interviews. \textit{Pragmatics} 12:257--271
% %
% % Contribution 
% \bibitem{humlinphil-contrib} Cameron, Deborah. 1997. Theoretical debates in feminist linguistics: Questions of sex and gender. In \textit{Gender and discourse}, ed. Ruth Wodak, 99--119. London: Sage Publications.
% %
% % Monograph
% \bibitem{humlinphil-mono} Cameron, Deborah. 1985. \textit{Feminism and linguistic theory.} New York: St. Martin's Press.
% %
% % Online Document
% \bibitem{humlinphil-online} Dod, Jake. 1999. Effective substances. In: The dictionary of substances and their effects. Royal Society of Chemistry. Available via DIALOG. \\
% http://www.rsc.org/dose/title of subordinate document. Cited 15 Jan 1999
%
% Journal article by DOI
% \bibitem{humlinphil-DOI} Suleiman, Camelia, Daniel C. O'Connell, and Sabine Kowal. 2002. `If you and I, if we, in this later day, lose that sacred fire...': Perspective in political interviews. \textit{Journal of Psycholinguistic Research}. doi: 10.1023/A:1015592129296.
%
%
%
% \bigskip
%
%
% Use the following syntax and markup for your references if 
% the subject of your book is from the field 
% "Computer Science, Economics, Engineering, Geosciences, Life Sciences"
%
%
% Contribution 
% \bibitem{basic-contrib} Brown B, Aaron M (2001) The politics of nature. In: Smith J (ed) The rise of modern genomics, 3rd edn. Wiley, New York 
% %
% % Online Document
% \bibitem{basic-online} Dod J (1999) Effective Substances. In: The dictionary of substances and their effects. Royal Society of Chemistry. Available via DIALOG. \\
% \url{http://www.rsc.org/dose/title of subordinate document. Cited 15 Jan 1999}
% %
% % Journal article by DOI
% \bibitem{basic-DOI} Slifka MK, Whitton JL (2000) Clinical implications of dysregulated cytokine production. J Mol Med, doi: 10.1007/s001090000086
% %
% % Journal article
% \bibitem{basic-journal} Smith J, Jones M Jr, Houghton L et al (1999) Future of health insurance. N Engl J Med 965:325--329
% %
% % Monograph
% \bibitem{basic-mono} South J, Blass B (2001) The future of modern genomics. Blackwell, London 
% %
%\end{thebibliography}
